%!TEX root = ../main.tex

\chapter{Risoluzione di sistemi lineari indeterminati}

Prendiamo un sistema lineare della forma:
\begin{equation*}
A\x =\mathbf{b}
\end{equation*}
dove $A\in \mathbb{R}^{m\times n} ,\mathbf{b} \in \mathbb{R}^{m} ,\x \in \mathbb{R}^{n}$. Il sistema si dice \textbf{indeterminato}\index{sistema!indeterminato} se $m\ne n$. In particolare:
\begin{itemize}
\item \textbf{sovradeterminato}\index{sistema!sovradeterminato} se $m >n$,
\item \textbf{sottodeterminato}\index{sistema!sottodeterminato} se $m< n$.
\end{itemize}

\section{Sistemi lineari sovradeterminati}
Approfondiremo ora i sistemi sovradeterminati, caso di maggiore interesse.
Sia dunque $A\in \mathbb{R}^{m\times n} ,m\geqslant n$.
Consideriamo il sistema lineare
\begin{equation}
A\x =\mathbf{b} ,\qquad\mathbf{b} \in \mathbb{R}^{m},
\label{eq:sist-lin-sovradet}
\end{equation}
osserviamo che \eqref{eq:sist-lin-sovradet} ha soluzione nel senso classico del termine solo se il termine noto è un elemento del seguente spazio:
\begin{equation*}
\operatorname{range}(A) =\left\{\y \in \mathbb{R}^{m} \ \text{tale che} \ A\x =\y \ \text{per qualche} \ \x \in \mathbb{R}^{n}\right\}.
\end{equation*}
Dobbiamo quindi allargare il concetto di \textit{soluzione} per studiare un sistema sovradeterminato.
\begin{definition}
[Soluzione di un sistema lineare nel senso dei minimi quadrati]
\index{soluzione!ai minimi quadrati}
Dati $A\in \mathbb{R}^{m\times n}$, $m\geqslant n$, e $\mathbf{b} \in \mathbb{R}^{m}$, diciamo che $\x^{\star } \in \mathbb{R}^{n}$ è soluzione di \eqref{eq:sist-lin-sovradet} nel senso dei minimi quadrati se
\begin{equation*}
\Phi \left(\x^{\star }\right) =\min_{\x \in \mathbb{R}^{n}} \Phi (\x),
\end{equation*}
dove il funzionale da minimizzare è $\Phi (\mathbf{w}) =\Vert A\mathbf{w} -\mathbf{b}\Vert ^{2}_{2}$, il residuo nella norma euclidea.
\end{definition}
\textit{Osservazione.} Se $m=n$, la soluzione $\x^{\star }$ nel senso dei minimi quadrati coincide con la soluzione classica perché $\Phi \left(\x^{\star }\right) =0$.

Dobbiamo chiarire se la soluzione ai minimi quadrati esista e se sia unica.
\begin{lemma}
Se la soluzione $\x^{\star }$ nel senso dei minimi quadrati di un sistema lineare esiste, essa coincide con la soluzione $\x^{\star }$ nel senso classico del \textit{sistema di equazioni normali}:
\begin{equation*}
A^{T} A\x^{\star } =A^{T}\mathbf{b}.
\end{equation*}
\end{lemma}
\textit{Dimostrazione.}
Scriviamo il funzionale da minimizzare:
\begin{align*}
\Phi (\x) & =\Vert A\x -\mathbf{b}\Vert ^{2}_{2}\\
 & =( A\x -\mathbf{b})^{T}( A\x -\mathbf{b})\\
 & =\left(( A\x)^{T} -\mathbf{b}^{T}\right)( A\x -\mathbf{b})\\
 & =( A\x)^{T} A\x-\underbrace{\mathbf{b}^{T}A\x}_{\langle \mathbf{b} ,A\x \rangle } -\underbrace{( A\x)^{T}\mathbf{b}}_{\langle A\x ,\mathbf{b} \rangle } +\mathbf{b}^{T}\mathbf{b}\\
 & =\x^{T} A^{T} A\x -2( A\x)^{T}\mathbf{b} +\mathbf{b}^{T}\mathbf{b}\\
 & =\x^{T} A^{T} A\x -2\x^{T} A^{T}\mathbf{b} +\mathbf{b}^{T}\mathbf{b}.
\end{align*}
Calcoliamone poi il gradiente
\begin{equation*}
\nabla \Phi (\x) =2A^{T} A\x -2A^{T}\mathbf{b}
\end{equation*}
e poniamolo uguale a zero per trovare il minimo:
\begin{gather*}
\nabla \Phi \left(\x^{\star }\right) =0\\
\Updownarrow \\
A^{T} A\x^{\star } -A^{T}\mathbf{b} =0\\
\Updownarrow \\
A^{T} A\x^{\star } =A^{T}\mathbf{b}.
\qed
\end{gather*}

\begin{theorem}
Sia $A\in \mathbb{R}^{m\times n}$, $m\geqslant n$.
Se $A$ ha rango pieno, cioè $\operatorname{rank}(A) =n$, allora $A^{T} A\in \mathbb{R}^{n\times n}$ è una matrice SDP e quindi il sistema di equazioni normali
\begin{equation*}
A^{T} A\x^{\star } =A^{T}\mathbf{b}
\end{equation*}
ammette una e una sola soluzione.
\end{theorem}

\textit{Osservazione.} La definizione di soluzione fornita è coerente con l'approssimazione nel senso dei minimi quadrati per dati e funzioni vista nella sezione \ref{sec:dati-min-quadrati}. Infatti per $\tilde {\boldsymbol  \alpha} =[\tilde \alpha_0, \tilde \alpha_1, \dots,\tilde \alpha_m]$ vale che 
\[
\tilde {\boldsymbol{\alpha}}=\operatornamewithlimits{argmin}_{\boldsymbol \alpha \in\mathbb R^{m+1}}\sum_{i=0}^n\big (y_i-(\alpha_0+\alpha_1x_i+\dots+\alpha_mx_i^m)\big)^2=\operatornamewithlimits{argmin}_{\boldsymbol \alpha \in\mathbb R^{m+1}}\|A \boldsymbol \alpha -\mathbf y\|_2
\]
Dove
\[
    A=\begin{bmatrix}
        1&x_0&\cdots&x_0^m\\
        \vdots&\vdots&\ddots&\vdots\\
        1&x_n&\cdots&x_n^m
    \end{bmatrix}\in \mathbb R^{n\times m+1},\quad \mathbf y=\begin{bmatrix}
        y_0\\
        \vdots\\
        y_m
    \end{bmatrix}.
\]
$\tilde{\boldsymbol{\alpha}}$ sarà la soluzione del sistema delle equazioni normali $A^TA\boldsymbol{\alpha}=A^T\mathbf y$, che ritroviamo essere lo stesso presentato nella sezione \ref{sec:caso-gen-min-quadrati}:
\[
\underbrace{\begin{bmatrix}
        1 & \cdots & 1 \\
        x_0 & \cdots & x_n \\
        \vdots & \ddots & \vdots \\
        x_0^m & \cdots & x_n^m
    \end{bmatrix}}_{A^T}
    \underbrace{\begin{bmatrix}
        1&x_0&\cdots&x_0^m\\
        \vdots&\vdots&\ddots&\vdots\\
        1&x_n&\cdots&x_n^m
    \end{bmatrix}}_A=\begin{bmatrix}( n+1) & \sum ^{n}_{i=0} x_{i} & \dotsc  & \sum ^{n}_{i=0} x^{m}_{i}\\
\sum ^{n}_{i=0} x_{i} & \sum ^{n}_{i=0} x^{2}_{i} & \dotsc  & \sum ^{n}_{i=0} x^{m+1}_{i}\\
\vdots  & \vdots  & \ddots  & \vdots \\
\sum ^{n}_{i=0} x^{m}_{i} & \sum ^{n}_{i=0} x^{m+1}_{i} & \dotsc  & \sum ^{n}_{i=0} x^{2m}_{i}
\end{bmatrix}
\]
e
\[
A^T\mathbf y=\begin{bmatrix}
        1 & \cdots & 1 \\
        x_0 & \cdots & x_n \\
        \vdots & \ddots & \vdots \\
        x_0^m & \cdots & x_n^m
    \end{bmatrix}\begin{bmatrix}
        y_0\\
        \vdots\\
        y_m
    \end{bmatrix}=\begin{bmatrix}
\sum ^{n}_{i=0} y_{i}\\
\sum ^{n}_{i=0} x_{i} y_{i}\\
\vdots \\
\sum ^{n}_{i=0} x^{m}_{i} y_{i}
\end{bmatrix}
\]

La regressione polinomiale, quindi, non è altro che un caso particolare di soluzione di un sistema nel senso dei minimi quadrati.

\textit{Osservazioni.}
\begin{itemize}
    \item Se $A^{T} A\x^{\star } =A^{T}\mathbf{b}$ ammette una e una sola soluzione in senso classico, allora \eqref{eq:sist-lin-sovradet} ammette una e una sola soluzione nel senso dei minimi quadrati.
    \item In genere, la matrice $A^{T} A$ è molto mal condizionata, quindi nella pratica è spesso difficile risolvere il sistema di equazioni normali per calcolare $\x^{\star }$.
    \item Se una matrice $A$ invertibile viene rappresentata in un calcolatore, a causa degli errori di arrotondamento è possibile che $A^TA$ perda l'invertibilità. Ad esempio,
    \[
        A = \begin{bmatrix}
            1 & 1 \\
            2^{-27} & 0 \\
            0 & 2^{-27}
        \end{bmatrix},\quad \operatorname{float}(A^T A)=\begin{bmatrix}
            1 & 1\\
            1 & 1
        \end{bmatrix}
    \]
\end{itemize}
Dobbiamo quindi trovare un altro modo di procedere per la risoluzione.
\subsection{Fattorizzazione QR}
Proviamo a generalizzare il concetto di fattorizzazione $LU$, visto nella sezione \ref{sec:fattorizzazione-lu} e proprio delle matrici quadrate, anche a matrici rettangolari come $A$.
\begin{definition}
[Fattorizzazione $QR$]
\index{fattorizzazione!QR}
Sia $A\in \mathbb{R}^{m\times n}$, $m\geqslant n$.
Si dice che $A$ ammette una fattorizzazione $QR$ se esistono:
\begin{itemize}
\item $Q\in \mathbb{R}^{m\times m}$ ortogonale ($Q^{-1} =Q^{T}$);
\item $R\in \mathbb{R}^{m\times n}$ trapezoidale superiore (con le righe dalla $n+1$ in poi tutte nulle) 
\end{itemize}
tali che $$A=QR$$
\end{definition}
\begin{equation*}
\underbrace{\begin{bmatrix}
\cdot  & \cdot  & \cdot \\
\cdot  & \cdot  & \cdot \\
\cdot  & \cdot  & \cdot \\
\cdot  & \cdot  & \cdot \\
\cdot  & \cdot  & \cdot
\end{bmatrix}}_{A\in \mathbb{R}^{m\times n}} =\underbrace{\begin{bmatrix}
\cdot  & \cdot  & \cdot  & \cdot  & \cdot \\
\cdot  & \cdot  & \cdot  & \cdot  & \cdot \\
\cdot  & \cdot  & \cdot  & \cdot  & \cdot \\
\cdot  & \cdot  & \cdot  & \cdot  & \cdot \\
\cdot  & \cdot  & \cdot  & \cdot  & \cdot
\end{bmatrix}}_{Q\in \mathbb{R}^{m\times m}}\underbrace{\begin{bmatrix}
\cdot  & \cdot  & \cdot \\
0 & \cdot  & \cdot \\
0 & 0 & \cdot \\
0 & 0 & 0\\
0 & 0 & 0
\end{bmatrix}}_{R\in \mathbb{R}^{m\times n}}
\end{equation*}
\textbf{Proprietà (Fattorizzazione }$QR$\textbf{ ridotta).} \index{fattorizzazione!QR!ridotta} Sia $A\in \mathbb{R}^{m\times n}$, $m\geqslant n$, di rango massimo di cui esista la fattorizzazione $QR$. Allora esiste un'unica fattorizzazione di $A$ tale che:
\begin{equation*}
A=\tilde{Q} \ \tilde{R}
\end{equation*}
dove $\tilde{Q}$ e $\tilde{R}$ sono le sottomatrici ottenute da $Q$ e $R$ nel seguente modo:
\begin{itemize}
\item $\tilde{Q} =Q( 1:m,1:n) \in \mathbb{R}^{m\times n}$;
\item $\tilde{R} =R( 1:n,1:n) \in \mathbb{R}^{n\times n}$.
\end{itemize}

Inoltre le colonne di $\tilde{Q}$ sono vettori ortonormali e $\tilde{R}$ coincide con il fattore di Cholesky della matrice $A^{T} A$, ossia $A^{T} A=\tilde{R}^{T}\tilde{R}$.
\begin{equation*}
\underbrace{\begin{bmatrix}
\cdot  & \cdot  & \cdot \\
\cdot  & \cdot  & \cdot \\
\cdot  & \cdot  & \cdot \\
\cdot  & \cdot  & \cdot \\
\cdot  & \cdot  & \cdot
\end{bmatrix}}_{A\in \mathbb{R}^{m\times n}} =\underbrace{\begin{bmatrix}
\begin{array}{|c c c|}
\hline
\cdot  & \cdot  & \cdot \\
\cdot  & \cdot  & \cdot \\
\cdot  & \tilde{Q} & \cdot \\
\cdot  & \cdot  & \cdot \\
\cdot  & \cdot  & \cdot \\
\hline
\end{array} & \begin{array}{ c }
\cdot \\
\cdot \\
\cdot \\
\cdot \\
\cdot
\end{array} & \begin{array}{ c }
\cdot \\
\cdot \\
\cdot \\
\cdot \\
\cdot
\end{array}
\end{bmatrix}}_{Q\in \mathbb{R}^{m\times m}}\underbrace{\begin{bmatrix}
\begin{array}{|c c c|}
\hline
\cdot  & \cdot  & \cdot \\
0 & \tilde{R} & \cdot \\
0 & 0 & \cdot \\
\hline
\end{array}\\
\begin{array}{ c c c }
0 & 0 & 0
\end{array}\\
\begin{array}{ c c c }
0 & 0 & 0
\end{array}
\end{bmatrix}}_{R\in \mathbb{R}^{m\times n}}
\end{equation*}
Come usiamo $A=\tilde{Q} \ \tilde{R}$ per risolvere $A\x =\mathbf{b}$?
\begin{theorem}
Sia $A\in \mathbb{R}^{m\times n}$, $m\geqslant n$, di rango massimo, e sia $\mathbf{b} \in \mathbb{R}^{m}$. Allora esiste un'unica soluzione $\x^{\star } \in \mathbb{R}^{n}$ nel senso dei minimi quadrati del sistema sovradimensionato $A\x =\mathbf{b}$ ed essa è data da:
\begin{equation*}
\x^{\star } =\tilde{R}^{-1}\tilde{Q}^{T}\mathbf{b} \qquad \text{ovvero} \qquad \tilde{R}\x^{\star } =\tilde{Q}^{T}\mathbf{b}
\end{equation*}
dove $\tilde{Q}$ ed $\tilde{R}$ sono i fattori della decomposizione ridotta di $A$.
\end{theorem}
\begin{equation*}
\underbrace{\begin{bmatrix}
\begin{array}{|c c c|}
\hline
\cdot  & \cdot  & \cdot \\
\cdot  & \cdot  & \cdot \\
\cdot  & \tilde{Q} & \cdot \\
\cdot  & \cdot  & \cdot \\
\cdot  & \cdot  & \cdot \\
\hline
\end{array} & \begin{array}{ c }
\cdot \\
\cdot \\
\cdot \\
\cdot \\
\cdot
\end{array} & \begin{array}{ c }
\cdot \\
\cdot \\
\cdot \\
\cdot \\
\cdot
\end{array}
\end{bmatrix}}_{Q\in \mathbb{R}^{m\times m}}\underbrace{\begin{bmatrix}
\begin{array}{|c c c|}
\hline
\cdot  & \cdot  & \cdot \\
0 & \tilde{R} & \cdot \\
0 & 0 & \cdot \\
\hline
\end{array}\\
\begin{array}{ c c c }
0 & 0 & 0
\end{array}\\
\begin{array}{ c c c }
0 & 0 & 0
\end{array}
\end{bmatrix}}_{R\in \mathbb{R}^{m\times n}}\underbrace{\begin{bmatrix}
\begin{array}{ c }
\cdot \\
\cdot \\
\cdot
\end{array}
\end{bmatrix}}_{\x \in \mathbb{R}^{n}} =\underbrace{\begin{bmatrix}
\begin{array}{ c }
\cdot \\
\cdot \\
\cdot \\
\cdot \\
\cdot
\end{array}
\end{bmatrix}}_{\mathbf{b} \in \mathbb{R}^{n}}
\end{equation*}
\textit{Idea di dimostrazione.} Ci concentreremo sulla dimostrazione dell'esistenza, tralasciando quella dell'unicità. Supponendo che $A$ ammetta una fattorizzazione $QR$, allora:
\begin{align*}
\Vert A\x -\mathbf{b}\Vert ^{2}_{2} & =\Vert QR\x -\mathbf{b}\Vert ^{2}_{2} & \\
 & =\left\Vert Q^{T} QR\x -Q^{T}\mathbf{b}\right\Vert ^{2}_{2} & \Vert \mathbf{z}\Vert _{2} =\left\Vert Q^{T}\mathbf{z}\right\Vert _{2} \ \forall \mathbf{z}\\
 & =\left\Vert R\x -Q^{T}\mathbf{b}\right\Vert ^{2}_{2} & \text{($Q$ è ortogonale)}\\
 & =\left\Vert \tilde{R}\x -\tilde{Q}^{T}\mathbf{b}\right\Vert ^{2}_{2} +\sum ^{m}_{i=n+1}\left[\left( Q^{T}\mathbf{b}\right)_{i}\right]^{2} & \forall \x \in \mathbb{R}^{n}.
\end{align*}
Osserviamo in particolare l'ultimo passaggio: abbiamo la norma (al quadrato, ma ciò è irrilevante) di un vettore, ovvero $R\x -Q^{T}\mathbf{b}$. Dato che la norma $2$ al quadrato è per definizione (vedi \ref{def:norma-vettore}) la somma delle componenti al quadrato, possiamo liberamente separare in due somme.
Il termine $R\x$ si separa in $\tilde{R}\x$ e una somma di zeri.
Il termine $Q^{T}\mathbf{b}$ si separa in $\tilde{Q}^{T}\mathbf{b}$ e il termine in sommatoria (si noti l'indice che va da $n+1$ a $m$).
Il minimo di $\Phi (\x)$ è raggiunto per $\x^{\star }$ che annulla tutto il termine in cui compare:
\begin{equation*}
\tilde{R}\x -\tilde{Q}^{T}\mathbf{b} =0\ \ \Rightarrow \ \ \left\Vert \tilde{R}\x -\tilde{Q}^{T}\mathbf{b}\right\Vert =0
\end{equation*}
e dunque $\x^{\star }$ soddisfa
\begin{gather*}
\tilde{R}\x^{\star } -\tilde{Q}^{T}\mathbf{b} =0\ \ \Rightarrow \ \ \x^{\star } =\tilde{R}^{-1}\tilde{Q}^{T}\mathbf{b}.
\qed
\end{gather*}
\textit{Osservazione.} Il costo di calcolare la fattorizzazione $QR$ è $\approx mn^{2}$. L'algoritmo per il calcolo di $Q$ è basato sull'algoritmo di Gram-Schmidt, la tecnica dall'algebra lineare per la costruzione di una base ortonormale.

\subsection{Sistemi sovradeterminati a rango non massimo}
Consideriamo ora un sistema sovradeterminato $A\x =\mathbf b$, dove $A\in \mathbb{R}^{m\times n},\ m>n$, ma $\operatorname{Rank(A)<n}$. La soluzione nel senso dei minimi quadrati non è più unica in quanto se $x^\star$ è tale da minimizzare $\Phi(x)$, anche $x^\star+z,\ \forall z\in \operatorname{Ker}(A)$ soddisfano la medesima condizione.

Occorre quindi porre un vincolo ulteriore per trovare una nozione che sia unica.

La soluzione che cerchiamo è, tra le soluzioni nel senso dei minimi quadrati, quella a norma minima.

Per trovare questa soluzione, faremo uso della \textbf{decomposizione ai valori singolari}\index{decomposizione!ai valori singolari} \footnote{in inglese \textit{SVD}, \textit{singular value decomposition}} che è una generalizzazione della decomposizione agli autovalori.

Ricordiamo che per una matrice $A\in \mathbb R^{n\times n}$ sufficientemente regolare si definisce la sua decomposizione agli autovalori, o decomposizione spettrale, data da
\begin{gather*}
    A=RDR^{-1},\quad  \text{dove} \quad D=\operatorname{diag}(\lambda_1,\dots,\lambda_n)\\R=[\x_1,\dots,\x_n], \quad \x_i=\text{autovettori di }A.
\end{gather*}

Cominciamo allora dalla definizione dei valori singolari.

\begin{definition}
    [Valori singolari]\index{valori singolari}
    Sia $A\in\mathbb R^{m\times n},\ m>n$. I suoi valori singolari sono le radici degli autovalori di $A^TA$, ovvero
    \[
    \sigma_i=\sqrt{\lambda _i(A^TA)},\quad i=1,\dots,n
    \]
\end{definition}

\begin{theorem}
    [Decomposizione ai valori singolari]
    Sia $A\in\mathbb R^{m\times n},\ m>n$. Allora $\exists U,V$ tali che 
    \[
    A = U\Sigma V^T,
    \]
    dove $U\in \mathbb R^{m\times m}$, $V\in \mathbb R^{n\times n}$, $\Sigma=\operatorname{diag}(\sigma_1,\dots,\sigma_n)$
\end{theorem}

\begin{equation*}
\underbrace{\begin{bmatrix}
\cdot  & \cdot  & \cdot \\
\cdot  & \cdot  & \cdot \\
\cdot  & \cdot  & \cdot \\
\cdot  & \cdot  & \cdot \\
\cdot  & \cdot  & \cdot
\end{bmatrix}}_{A\in \mathbb{R}^{m\times n}} =\underbrace{\begin{bmatrix}
\cdot  & \cdot  & \cdot  & \cdot  & \cdot \\
\cdot  & \cdot  & \cdot  & \cdot  & \cdot \\
\cdot  & \cdot  & \cdot  & \cdot  & \cdot \\
\cdot  & \cdot  & \cdot  & \cdot  & \cdot \\
\cdot  & \cdot  & \cdot  & \cdot  & \cdot
\end{bmatrix}}_{U\in \mathbb{R}^{m\times m}}\underbrace{\begin{bmatrix}
\sigma_1  & 0  & 0 \\
0 & \ddots  & 0 \\
0 & 0 & \sigma_n \\
0 & 0 & 0\\
0 & 0 & 0
\end{bmatrix}}_{\Sigma\in \mathbb{R}^{m\times n}}
\underbrace{\begin{bmatrix}
    \cdot  & \cdot  & \cdot \\
    \cdot  & \cdot  & \cdot \\
    \cdot  & \cdot  & \cdot \\
\end{bmatrix}}_{V^T\in\mathbb R^{n\times n}}
\end{equation*}
\textit{Osservazione.} Il costo di calcolare la decomposizione \textit{SVD} è $\approx 12 n^3$.

\begin{definition}
    [Matrice pseudoinversa]
    Siano $A,U,V$ definiti come nel teorema. Definiamo la matrice pseudoinversa di $A$
    \[
    A^\dagger =V\Sigma^\dagger U^T,\quad \text{dove} \ \Sigma^\dagger =\operatorname{diag}\left(\frac1{\sigma_1},\dots,\frac1{\sigma_n}\right)^T
    \]
\end{definition}

\begin{theorem}
    Sia $A\in \mathbb{R}^{m\times n},\ m>n$, $\operatorname{Rank(A)<n}$. La soluzione $\x^\star$ nel senso dei minimi quadrati del sistema $A\x =\mathbb b$ di norma minima soddisfa
    \[
    \x^\star =A^\dagger\mathbf  b
    \]
\end{theorem}

\section{Sistemi lineari sottodeterminati}

Trattiamo brevemente anche i sistemi sottodeterminati. Sia $A\in \mathbb R^{m\times n}$, $m<n$. Consideriamo il sistema lineare
\[
A\x =\mathbf b,\quad \mathbf b\in \mathbb R^m.
\]
Per il teorema di Rouché-Capelli la soluzione del sistema non è unica. Procediamo quindi cercando la soluzione che soddisfi una condizione aggiuntiva, cioè che abbia norma minima.
\begin{theorem}
    Siano dati $A\in \mathbb R^{n\times m},\ m<n,\ \operatorname{Rank}(A)=m,\ \mathbf b\in \mathbb R^m$. $\x^\star = A^T(AA^T)^{-1}\mathbf b$ soddisfa le seguenti proprietà:
    \begin{gather*}
        A\x^\star =\mathbf b\tag{1}\label{eq:sottodet-soluz}\\
        \x^\star = \operatornamewithlimits{argmin}_{\x \in\mathbb R^n:A\x=\mathbf b} \|\x\| \tag{2}\label{eq:sottodet-min-norma}\\
    \end{gather*}
\end{theorem}

\textit{Dimostrazione.}

Per \eqref{eq:sottodet-soluz} è sufficiente notare $A\x^\star = AA^T (AA^T)^{-1}\mathbf b=\mathbf b$.

Per \eqref{eq:sottodet-min-norma}, dimostriamo che $\|\x^\star\|^2\le \|\x\|^2\ \forall \x\in\mathbb R^n$ tale che $A\x =\mathbf b$.

Sia dunque $\x$ tale che $A\x = \mathbf b$.
\begin{multline*}
    \|\x\|^2=\|\x^\star +(\x -\x^\star)\|=(\x^\star +(\x -\x^\star))^T(\x^\star +(\x -\x^\star))=\\=\|\x^\star \|^2+\|\x-\x^\star\|^2+2\underbrace{(\x -\x^\star)^T\x^\star}_{\triangle}
\end{multline*}

Osserviamo che 
\[
\triangle=\underbrace{(\x-\x^\star)^TA^T}_{\left(A(\x-\x^\star)\right)^T}(AA^T)^{-1}\mathbf b
\]
Ma $A(\x-\x^\star)=A\x-A\x^\star =\mathbf b-\mathbf b=\mathbf 0$, quindi riprendendo l'equazione sopra
\[
\|\x\| = \|\x^\star \|^2+\underbrace{\|\x-\x^\star\|^2}_{\ge 0} + \mathbf 0\ge \|\x^\star \|^2\qed
\]

\textit{Osservazione.} La matrice $A^T(AA^T)^{-1}$ è fortemente mal condizionata, dunque è necessario procedere in modo alternativo per determinare la soluzione. Usiamo la fattorizzazione $QR$ ridotta di $A^T$. Sia dunque $A^T=\tilde Q\tilde R$. Allora riscriviamo $\x^\star$ come
\begin{multline*}
    \x^\star = A^T(AA^T)^{-1}\mathbf b = (\tilde Q \tilde R)(( \tilde Q \tilde R)^T( \tilde Q \tilde R))^{-1}\mathbf b=\\=
    (\tilde Q \tilde R)(\tilde R^T \underbrace{\tilde Q^T \tilde Q}_I \tilde R)^{-1}\mathbf b = 
    \tilde Q \underbrace{\tilde R  \tilde R^{-1}}_I(\tilde R^T)^{-1} \mathbf b = 
    \tilde Q (\tilde R^T)^{-1} \mathbf b
\end{multline*}
